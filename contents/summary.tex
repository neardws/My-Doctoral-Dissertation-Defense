\section[\englishfont 3 总结与展望]{总结与展望}
\begin{frame}{总结与展望} 
\newBackground
\begin{center}
\begin{textblock*}{1.2\textwidth}(1cm,2cm)
\begin{itemize}[itemsep=0.2\baselineskip]
 \englishfont
	\item[\ding{111}] {\color{cqublue}{工作总结}}
	\begin{itemize}[itemsep=0.2\baselineskip]
	\begin{small}
		\item[\ding{226}] \underline{架构}: 车联网分层服务架构,{\color{cqublue}{有机融合SDN与MEC}}
		\item[\ding{226}]  \underline{指标}: 车载信息物理融合质量指标,考虑{\color{cqublue}{时效性、完整性和一致性}}
		\item[\ding{226}] \underline{MADR算法}: 有效{\color{cqublue}{提高VCPS质量}}
		\item[\ding{226}] \underline{CRO问题}:协同传输与计算,以最大化任务完成率
		\item[\ding{226}] \underline{MAGT算法}:有效{\color{cqublue}{提高任务完成率}}
		\item[\ding{226}] \underline{双目标优化问题}:车载信息物理融合系统{\color{cqublue}{质量模型和开销模型}}
		\item[\ding{226}] \underline{MAMO算法}:有效实现{\color{cqublue}{质量和开销的均衡}}
	\end{small}
	\end{itemize}
	\item[\ding{111}]  {\color{cqublue}{研究展望}}
	\begin{itemize}[itemsep=0.2\baselineskip]
	\begin{small}
		\item[\ding{226}] 边缘节点之间的合作
		\item[\ding{226}] 车联网端边云架构,车辆、边缘节点和云协同
		\item 
	\end{small} 
	\end{itemize}
\end{itemize}
\end{textblock*}
\end{center}
\end{frame}
